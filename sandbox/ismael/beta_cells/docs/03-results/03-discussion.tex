In cell migration, shape of the culture substrate and cell density are main factors to consider when results are analyzed. Cells in most of tissues need to create structures in three dimensions to achieve functionality. Usually, they tend to present a compact collective geometry to minimize space required in the organism.\\
\\
Cluster variable mean represented versus time steps in both 2D culture and micro-encapsulation, 3D culture example is not considered in this comparison due to its lower cellular density that could alter our conclusions. Cell clusters are more compact in the 3D environment, cluster variable mean is higher in micro-encapsulation simulation even with a lower initial values and there is also a notably faster cluster growing.\\
\\
Diffusion phenomenon in the examples will be affected by ECM parameters, nutrient concentration in the media and cell presence. Two first parameter are fixed in the model and these do not suffer great changes either in in vitro experiments, ECM intrinsic characteristics remains almost constant and media is usually renewed to provide culture fresh nutrients. Cell presence will determine local diffusion constant in the ECM, reducing it dramatically, and it is also need to consider nutrient consumption by cells.\\
\\
Firstly, nutrient reaches cells almost instantly in 2D culture and keep its concentration at the same value that media, in this example media is in direct contact with cells all along the culture plate. Secondly, in the scaffold example this process is notably slower as it is said before due to its dimensions. Nevertheless, behavior of nutrient in micro-encapsulation example falls in between of both previous examples, despite not being instant, concentration grows much faster than it does in scaffold, although it do not reach same nutrient concentration that media as in 2D example.\\
\\
Nutrient availability is fundamental to cell survival, therefore, a poor nutrient distribution in the ECM will drive to abnormal cell development or death due to starvation. In this model, only nutrient concentration is taken in account to determine if a cell is viable but other factors may also influence results.\\
\\
On one hand, both 2D culture and micro-capsule have not noticeable loss in the global cell survivability , this is a logical result knowing that nutrient concentration is homogeneous as it was previously said and is also constant in the medium. On the other hand, 3D culture scaffold shows a significant health reduction until approximately step 125, this is a direct consequence of time that takes to the nutrient to reach an acceptable concentration at scaffold. After that, there is a recovery to a value lower than initial global health score, this is a sign that some cells have died in the model and their health score is zero reducing global health score permanently.\\
\\
Therefore, poor nutrient diffusion has a strong effect on cell viability. If there is a need of bigger cultures, this problem may be solved in in vitro experimentation by the use of some advanced techniques of cultivation as perfusion bioreactors (\cite{Dunn2006}, \cite{Goh2013}) or micro-encapsulation.