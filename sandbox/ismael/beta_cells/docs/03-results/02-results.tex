\subsubsection{2D Culture Plate}
As it is seen in Figure \ref{fig:2d-glyphs} structures formed in 2D cultures are limited to its shape. At an early stage of simulation, cells are uniformly distributed through the plate and they tend to form cell aggregates during the simulation run. Additionally, cell patterns can be observed at later steps forming big cell clusters and also some individual cells that are out of the influence of other cells attraction.\\ 
\\
Cluster variable in the program is calculated as cells that are near to the current cell, proximity consideration is set proportional to its radius. In Figure \ref{fig:2d-table-cluster}, it is shown cluster variable mean during simulation time for 2D culture, a noticeable increment can be appreciated.\\
\\
Finally, nutrient concentration measured in the middle of the culture ECM shows an almost instant nutrient equilibrium (Figure \ref{fig:2d-table-conc}), it reaches the medium concentration value. Being height of the substrate negligible, cells in this culture can be considered in direct contact with the medium.

\begin{figure}[H]
\centering
\setlength\fboxsep{0pt}
\setlength\fboxrule{0.5pt}
\includegraphics[width=\textwidth]{img/2d/glyphs}
\caption{Cluster Formation in 2D culture plate simulation. Time Steps: (a) 0 (b) 15 (c) 250 (d) 500}
\label{fig:2d-glyphs}
\end{figure}

\begin{figure}[H]
\centering
\setlength\fboxsep{0pt}
\setlength\fboxrule{0.5pt}
\includegraphics[width=\textwidth]{img/2d/2d-table-cluster}
\caption{Cluster variable mean in 2D culture.}
\label{fig:2d-table-cluster}
\end{figure}

\begin{figure}[H]
\centering
\setlength\fboxsep{0pt}
\setlength\fboxrule{0.5pt}
\includegraphics[width=\textwidth]{img/2d/2d-table-conc}
\caption{Nutrient concentration in the 2D culture at the volume center.}
\label{fig:2d-table-conc}
\end{figure}

\newpage
\subsubsection{3D Culture Scaffold}
In the 3D culture scaffold representation, only cells with cluster variable value above a threshold (set as 5) are shown in Figure \ref{fig:3d-glyphs} to make results visualization clearer. It is observed that in early stages there are only very few groups of cells, more clusters will appear in later stages. Due to the big scaffold size and high number of cells, in this example is difficult to see the structures that cells have formed but they are a lot of like to those that are shown in the micro-encapsulation example.\\ 
\\
Nutrient diffusion in this example is absolutely not negligible, nutrient do not reach scaffold interior instantly. Nutrient distribution is not uniform in the scaffold as in 2D cultures during the experimentation, therefore, cells near boundaries will be more likely to receive better nutrition than the ones near scaffold center. Even in later stages, nutrient concentration does not seem to be well distributed in the scaffold as it is shown in Figure \ref{fig:3d-conc}.\\
\\
Figure \ref{fig:3d-table-conc} shows that nutrient concentration inside scaffold slowly increases and do not reach medium concentration. It seems that does not even reach equilibrium with the steps used in this simulation. Therefore, size of the scaffold is a limiting factor to nutrient diffusion through culture.\\
\\
Considering the viability model used, a lack of nutrient during a given time is the only cause of cell death. In Figure \ref{fig:3d-table-health} is shown how global health is affected by the low concentration of nutrients in some scaffold areas. Initially, global health is reduced to almost 80\% followed by an increment up to 96\%. This permanent decrease of global cell survivability variable is caused by death of some cells, mostly near scaffold center. In Figure \ref{fig:3d-dead}, there is a representation of dead cells in the 3D culture scaffold due to starvation.

\begin{figure}[H]
\centering
\setlength\fboxsep{0pt}
\setlength\fboxrule{0.5pt}
\includegraphics[width=\textwidth]{img/3d/glyphs}
\caption{Cluster Formation in 3D culture scaffold simulation. Time Steps: (a) 0 (b) 15 (c) 250 (d) 500}
\label{fig:3d-glyphs}
\end{figure}

\begin{figure}[H]
\centering
\setlength\fboxsep{0pt}
\setlength\fboxrule{0.5pt}
\includegraphics[width=300pt]{img/3d/conc}
\caption{Diffusion in 3D culture scaffold. Time steps: (a) 0 (b) 15 (c) 250 (d) 500}
\label{fig:3d-conc}
\end{figure}

\begin{figure}[H]
\centering
\setlength\fboxsep{0pt}
\setlength\fboxrule{0.5pt}
\includegraphics[width=\textwidth]{img/3d/3d-table-conc}
\caption{Nutrient concentration in the 3D culture scaffold at the volume center.}
\label{fig:3d-table-conc}
\end{figure}

\vspace{15mm}

\begin{figure}[H]
\centering
\setlength\fboxsep{0pt}
\setlength\fboxrule{0.5pt}
\includegraphics[width=\textwidth]{img/3d/3d-table-health}
\caption{Cell survivability mean in 3D culture scaffold.}
\label{fig:3d-table-health}
\end{figure}

\begin{figure}[H]
\centering
\setlength\fboxsep{0pt}
\setlength\fboxrule{0.5pt}
\includegraphics[width=\textwidth]{img/3d/3d-dead}
\caption{Dead cells in 3D culture scaffold at last step.}
\label{fig:3d-dead}
\end{figure}

\newpage
\subsubsection{Micro-encapsulation}
In micro-encapsulation, cell environment can be considered as a scaffold like common 3D cultures but much smaller. Figure \ref{fig:micro-glyphs} shows the formation of very compact clusters of cells in later stages of simulation. It can be appreciated four clusters that are connected by some cells but they are not tending to form one big cell aggregate.\\
\\
Figure \ref{fig:micro-table-cluster} shows an increment of the cluster variable mean in this kind of culture that almost reach 600\% of the initial value. Cells in micro-capsules can migrate and form 3D clusters that are much more compact than 2D structures.\\
\\
Micro-capsule seems to reach a better nutrient distribution. Its lower size allows this 3D environment to surpass scaffold main problem for cells and tissues development. As it was shown in Figure \ref{fig:micro-conc}, several compact clusters are created inside the micro-capsule, consequently, there are regions of higher nutrient consumption and lower nutrient diffusion due to massive cell presence that locally reduces nutrient concentration.\\ 
\\
High diffusion efficiency can be appreciated in Figure \ref{fig:micro-table-conc}, nutrient concentration in the middle of the capsule reaches a constant concentration value slightly lower than nutrient concentration in the medium. 

\begin{figure}[H]
\centering
\setlength\fboxsep{0pt}
\setlength\fboxrule{0.5pt}
\includegraphics[width=\textwidth]{img/3dm/glyphs}
\caption{Cluster Formation in Micro-encapsulation simulation. Time Steps: (a) 0 (b) 15 (c) 250 (d) 500}
\label{fig:micro-glyphs}
\end{figure}

\begin{figure}[H]
\centering
\setlength\fboxsep{0pt}
\setlength\fboxrule{0.5pt}
\includegraphics[width=\textwidth]{img/3dm/micro-table-cluster}
\caption{Cluster variable mean in Micro-encapsulation.}
\label{fig:micro-table-cluster}
\end{figure}

\begin{figure}[H]
\centering
\setlength\fboxsep{0pt}
\setlength\fboxrule{0.5pt}
\includegraphics[width=\textwidth]{img/3dm/micro-table-conc}
\caption{Nutrient concentration in the Micro-encapsulation at the volume center.}
\label{fig:micro-table-conc}
\end{figure}

\begin{figure}[H]
\centering
\setlength\fboxsep{0pt}
\setlength\fboxrule{0.5pt}
\includegraphics[width=\textwidth]{img/3dm/conc}
\caption{Diffusion in Micro-encapsulation. Time steps: (a) 0 (b) 15 (c) 250 (d) 500}
\label{fig:micro-conc}
\end{figure}





\newpage   