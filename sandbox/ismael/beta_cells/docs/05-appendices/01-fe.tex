To understand complex systems is useful to focus on the key aspects and try to use them to achieve a simplified representation. Obtaining a model will allow to study more closely behavior of the system and also to make predictions with different input parameters. Modelling techniques are widely used in a wide range of scientific and engineering fields. \textit{Finite elements method} (\textbf{FEM}) is a numerical approach used to solve differential equation of the models in situations where classical analytical methods are not able due the inherent complexity.\\ 
\\
This method is commonly used in stress calculation, heat transfer, mass transport, fluid flow or electromagnetic. Recently, this method also is being used to simulate  biological systems, mainly to study biomechanics and mechanobiology. FEM relies on computers to find approximate solutions numerically and usually is compared to physical models to confirm its validity.\\
\\
In \textbf{FEM}, the basic idea is to divide a volume into simpler elements connected by nodes, this allow to obtain approximate solution in complex shapes. The body division is called mesh and the process to obtain it mesh generation. In Figure \ref{fig:mesh} is shown a mesh of a simple geometry to illustrate this procedure.
 
\begin{figure}[H]
\centering
\setlength\fboxsep{0pt}
\setlength\fboxrule{0.5pt}
\includegraphics[width=120pt]{img/cubemesh}
\caption{Cube mesh example.}
\label{fig:mesh}
\end{figure}

For linear problems, solution is achieved by solving the model equations considering that the number of unknowns are equal to the number of nodes. Usually, it used thousands of elements to obtain accurate solutions, increasing both accuracy and computational cost with the number of nodes used.\\
\\
This method allows to study a wide range of phenomena in one program and, in consequence, highly complex problems. For example, in Figure \ref{fig:complexmesh} is shown a complex shape where solving deformation by stress and heat flow at the same time by analytical methods would be practically impossible. In contrast, this calculation using a properly \textbf{FEM} implementation is simply solved.\\
\\

\begin{figure}[H]
\centering
\setlength\fboxsep{0pt}
\setlength\fboxrule{0.5pt}
\includegraphics[width=250pt]{img/complexmesh}
\caption{Complex object mesh example.}
\label{fig:complexmesh}
\end{figure}

Finally, there is a very large number of applications of this method on real life problems, but we list several of them to illustrate its versatility:

\begin{enumerate}
\item Industrial parts analysis focusing stress and heat transfer (pipes, reactors, vessels, engines, airplane components,...)

\item Analysis of seismological effects on buildings, power plants or large structures.

\item Forensic analysis of transport accidents.

\item Analysis of fluid flow in heat exchangers, destilators or cooling equipment.

\item Analysis of clinical procedures as bones reconstruction, plastic surgery or dental implants.
\end{enumerate}

