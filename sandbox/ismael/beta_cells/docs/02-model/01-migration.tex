We propose a discrete model to calculate cell-cell interaction and cell drag in the scaffold. This model will allow predicting cell movement and track cluster formation, being able to represent both individual and collective migration.\\
\\
Force equilibrium on the cell centroid as is seen in equation \eqref{eq:equilibrium} dictates cell movement. Cells are considered spherical and number of cells is represented in the equation by N parameter. In this model, there are two kinds of forces exerted on the cells: cell-cell interaction forces and cell drag force \footnote{As a first approach, we do not consider Cell-ECM interaction, that could be easily included as for example in \cite{Rey2013}.}. 
\\
\begin{equation}  \label{eq:equilibrium}
\sum\limits_{j=1}^N (F_{ji,c-c})+ F_{i, drag}=0 \qquad\qquad \forall_i=1,...,N
\end{equation}
\\
Cell to cell interaction forces are calculated by a modified Morse potential equation \eqref{eq:morse}. Therefore, this interaction can be approximated to particle-particle interaction, where two cells are attracted until a certain distance where they will repulse each other. The attractive actions correspond to mechano-chemical signaling between cells and the chemical bonds created in their membranes when they are close. Repulsive actions represent cytoskeleton and cytoplasm stiffness, occurring when they are occupying same space.\\
\\
This interactive force includes chemical and mechanical attraction and it is calculated in binary combinations of all cells in the pool. Four parameters are used to calculate forces (A, B, $\epsilon_1$, $\epsilon_2$) and both cell radius ($r_i$ and $r_j$)as is explained by \cite{Rey2013}.

\begin{equation}  \label{eq:morse}
F_{ji,c-c}=\left(A*exp\left( -\dfrac{||r_j-r_i||}{\varepsilon_1} \right) + B*exp\left( -\dfrac{||r_j-r_i||}{\varepsilon_2} \right) \right) * \dfrac{r_j-r_i}{||r_j-r_i||}
\end{equation}


A resistance term is added to the cell force calculation due to the difficult movement through the cavities of the ECM and the force drag. This resistance is considered proportional and opposed to the velocity of the cell (\cite{Zaman2005}), $f_d$ parameter is equal to the viscous drag of a sphere in a infinitely viscous medium. This assumption does not consider the shape or size of cells and the scaffold local polymeric structure heterogenous.

\begin{equation}  \label{eq:drag}
F_{i,drag}=-f_d*v_i 
\end{equation}

Equation \eqref{eq:drag} is used to obtain cell velocity in the current step knowing total force equilibrium. Consequently, position on the next step is calculated with known position of the cell and constant velocity supposed in that frame of time.

