\documentclass[a4paper,DIV=12,10pt]{scrartcl}

\usepackage{notes}

\usepackage{color}
\usepackage{booktabs}
\usepackage{comment}
\usepackage{tikz}

\graphicspath{{./pics/}}

%%------------------------------------------------------------------------------
\newcommand{\U}[0]{\vek{u}}
\newcommand{\V}[0]{\vek{v}}
\newcommand{\x}[0]{\vek{x}}


%%------------------------------------------------------------------------------
\begin{document}

\workingnote{A simplified multi-phasic material}{September 17, 2013}


Derivations are based on~\cite{ateshian2006mixture}, \cite{zhang2007effect},
\cite{boonkkamp}, 
and~\cite{bowen1980incompressible}.

% ------------------------------------------------------------------------------
\section{Mixture theory and notation}
\label{sec:mixture}
Let $\alpha$ label a constituent in a mixture of $N$ constituents. Moreover,
$\alpha = s$ is the solid part and $\alpha = f$ the fluid part. Any other
$\alpha \neq f, s$ denotes, for the time being, a solute which is contained in
the fluid phase.
Now, $\alpha$ has mass $m^\alpha$ and occupies a volume $V^\alpha$.  The total
mass is $m = \sum m^\alpha$ and the total volume $V = \sum V^\alpha$. With
these definitions, the true $\rho^\alpha_T$ and apparent density
$\rho^\alpha$, and the volume fraction $\phi^\alpha$ can be defined and
related 
\begin{equation}
  \label{eq:density}
  \begin{aligned}
    \phi^\alpha   &= \frac{V^\alpha}{V} \\
    \rho^\alpha_T &= \frac{m^\alpha}{V^\alpha} \\
    \rho^\alpha   &= \frac{m^\alpha}{V} = \rho^\alpha_T \phi^\alpha\,.
  \end{aligned}
\end{equation}
Let $\rho = \frac{M}{V} = \sum \rho^\alpha$ denote the average mixtures
density and, thus, $m^\alpha / m = \rho^\alpha / \rho$ is the mass ratio of
$\alpha$.

% ------------------------------------------------------------------------------
\section{Momentum balances}
\label{sec:momentum}

\paragraph{Mixture}
\begin{equation}
  \label{eq:momMix}
  -\nabla \cdot \vek{\sigma}(\U) + \nabla p = \vek{f}
\end{equation}
with the elastic stress $\vek{\sigma}$ in function of the solid displacement
$\U$ and the fluid pressure $p$.

\paragraph{Solvent and solute}
\begin{align}
  \label{eq:momSolvent}
  - \rho^f \nabla \mu^f + \sum_{\beta} f_{f \beta} (\V^\beta  - \V^f) &=
  \vek{0} \\
  \label{eq:momSolute}
  - \rho^\alpha \nabla \mu^\alpha + \sum_{\beta} f_{\alpha \beta} (\V^\beta
  - \V^\alpha) &=  \vek{0} 
\end{align}
with the apparent density $\rho^\alpha$, the velocities of the constituents
$\V^\alpha$, the diffusive drag coefficients $f_{\alpha \beta}$ and the
chemical potentials
\begin{align}
  \label{eq:chemPotSolvent}
  \mu^f &= \mu^f_0 + \frac{1}{\rho^f_T} \left( p - RT \sum_{\beta}
    c^\beta \right)\\
  \label{eq:chemPotSolute}
  \mu^\alpha &= \mu^\alpha_0 + \frac{RT}{M^\alpha} \log
  \frac{c^\alpha}{\kappa^\alpha c^\alpha_0}\,.
\end{align}
Here, the chemical potentials $\mu_0$ of the standard state are functions of
the temperature $T$ only and their gradients vanish. $\rho^f_T$ is the true
fluid density, $R$ the universal gas constant, $M^\alpha$ the molecular
weight, and $\kappa^\alpha$ a partition coefficient. Most importantly, 
$c^\alpha$ is the concentration of the solute $\alpha$ based on the solvent
volume. 
The sums in equations~\eqref{eq:momSolvent} and~\eqref{eq:momSolute} run over
all phases including the solid $\beta = s$ and the fluid $\beta = f$. The sum
in equation~\eqref{eq:chemPotSolvent} does not incorporate the cases $\beta
= s$ and $\beta =f$, as the concentrations $c^s$ and $c^f$ do not make sense.


The substitution of~\eqref{eq:chemPotSolvent} into~\eqref{eq:momSolvent}
yields, when identifying the fluid volume fraction $\phi^f = \rho^f /
\rho^f_T$, 
\begin{equation}
  - \phi^f \left( \nabla p - RT \sum_{\beta} \nabla c^\beta \right) +
  \sum_{\beta} f_{f\beta} (\V^\beta - \V^f) = \vek{0}
\end{equation}
The apparent density of the solute $\alpha$ can be identified as 
$\rho^\alpha = M^\alpha \phi^f c^\alpha$ and
substituting~\eqref{eq:chemPotSolute} into~\eqref{eq:momSolute} gives
\begin{equation}
  - \phi^f RT \nabla c^\alpha + \sum_{\beta} f_{\alpha \beta} 
  (\V^\beta -  \V^\alpha) = \vek{0} \,.
\end{equation}

Now the diffusive drag coefficients have to be specified. We have
$f_{\alpha\beta} = f_{\beta\alpha}$ and, moreover, assume that the drag
between the solutes is negligible, i.e. $f_{\alpha\beta} = 0$ for $\alpha \neq
s, f$ and $\beta \neq s, f$. The remaining coefficients for drag with respect
to the fluid constituent $\beta = f$ are
\begin{equation}
  \label{eq:fluidDrag}
  f_{sf} = \frac{ (\phi^f)^2 }{k} \quad \text{and} \quad
  f_{\alpha f} = \frac{\phi^f RT c^\alpha}{D^\alpha_0}
\end{equation}
with the hydraulic permeability $k$ and the diffusivity in free solution
$D^\alpha_0$. 
Based on these expressions, the fluid momentum balance becomes
\begin{equation}
  - \nabla p + RT \sum_{\beta} \nabla c^\beta + \frac{\phi^f}{k} (\V^s - \V^f)
  + \sum_{\beta\neq s} \frac{RT c^\beta}{D_0^\beta} (\V^\beta - \V^f) = \vek{0}\,.
\end{equation}

The momentum balance of the solute $\alpha$, on the other hand is re-expressed
as
\begin{equation}
  \begin{aligned}
    -\phi^f RT \nabla c^\alpha + f_{\alpha s}(\V^s - \V^\alpha) 
    + f_{\alpha f} (\V^f - \V^\alpha) &= \\
    -\phi^f RT \nabla c^\alpha + f_{\alpha s}(\V^s - \V^\alpha) 
    + f_{\alpha f} (\V^s - \V^\alpha + \V^f - \V^s) &= \\
    -\phi^f RT \nabla c^\alpha + (f_{\alpha s} + f_{\alpha f} )(\V^s - \V^\alpha) 
    + f_{\alpha f} (\V^f - \V^s) &= \vek{0} 
   \end{aligned}
\end{equation}
The drag coefficient $f_{\alpha f}$ is already given in~\eqref{eq:fluidDrag}
and we define the sum 
\begin{equation}
  \label{eq:modDrag}
  f_{\alpha s} +  f_{\alpha f} = \frac{\phi^f RT c^\alpha}{D^\alpha}
\end{equation}
in a similar fashion, but now using the diffusivity $D^\alpha$ of the solute
in mixture, which is in general a value smaller than $D^\alpha_0$.
Based on this definition, the momentum balance of constituent $\alpha$ becomes
\begin{equation}
  - \nabla c^\alpha + \frac{c^\alpha}{D^\alpha}(\V^s - \V^\alpha) 
  + \frac{c^\alpha}{D^\alpha_0} (\V^f - \V^s) = \vek{0} \,.
\end{equation}

From this last equation, we get the expression
\begin{equation} \label{eq:aux1}
  \V^\beta - \V^s = \frac{D^\beta}{D^\beta_0} (\V^f - \V^s) -
  \frac{D^\beta}{c^\beta} \nabla c^\beta\,.
\end{equation}
Now the fluid momentum balance is reconsidered in order to eliminate the
relative velocities $\V^\beta - \V^f$ based on expression~\eqref{eq:aux1}
\begin{equation}
  \begin{aligned}
    - \nabla p + RT \sum_{\beta} \nabla c^\beta + \frac{\phi^f}{k} (\V^s - \V^f)
    + \sum_{\beta\neq s} \frac{RT c^\beta}{D_0^\beta} (\V^\beta - \V^f) & = \\  
    - \nabla p + RT \sum_{\beta} \nabla c^\beta + \frac{\phi^f}{k} (\V^s - \V^f)
    + \sum_{\beta\neq s} \frac{RT c^\beta}{D_0^\beta} 
    \left[(\V^\beta - \V^s) + (\V^s - \V^f) \right] &=  \\
    -\nabla p + RT \sum_{\beta} \left(1 - \frac{D^\beta}{D^\beta_0} \right)
    \nabla c^\beta +
    \left[ \frac{\phi^f}{k} + RT \sum_{\beta} 
      \left(1 - \frac{D^\beta}{D^\beta_0} \right) 
      \frac{c^\beta}{D^\beta_0} \right] (\V^s - \V^f) &= \\
    -\nabla p + RT \sum_{\beta} \left(1 - \frac{D^\beta}{D^\beta_0} \right)
    \nabla c^\beta +
    \frac{\phi^f}{k_{\text{eff}}} (\V^s - \V^f) &=0\,.
  \end{aligned}
\end{equation}
Here, an effective permeability $k_{\text{eff}}$ has been introduced which is
defined via
\begin{equation}
  \label{eq:effectivePermeability}
  \frac{1}{ k_{\text{eff}} } = \frac{1}{k} + \frac{RT}{\phi^f} 
  \sum_{\beta} \left(1 - \frac{D^\beta}{D^\beta_0} \right) \frac{c^\beta}{D^\beta_0} \,.
\end{equation}
With this simplification, the final momentum balance leads to the volumetric
flux (sometimes referred to as Darcy velocity) 
\begin{equation}
  \label{eq:volumetrixFlux}
  \vek{j}_{\text{vol}} = \phi^f (\V^f - \V^s) = - k_{\text{eff}} \left[ \nabla p - 
    RT \sum_{\beta} \left(1 - \frac{D^\beta}{D^\beta_0} \right)
    \nabla c^\beta \right]\,.
\end{equation}
On the other hand, re-writing the momentum balance of the constituent $\alpha$
gives the definition of the molecular flux of which
\begin{equation}
  \label{eq:molecularFlux}
  \vek{j}^\alpha = \phi^f c^\alpha (\V^\alpha - \V^s) 
  = - \phi^f D^\alpha \nabla c^\alpha 
  + \frac{D^\alpha}{D^\alpha_0} c^\alpha \phi^f ( \V^f - \V^s )
  = -\phi^f D^\alpha \nabla c^\alpha 
  + \frac{D^\alpha}{D^\alpha_0} c^\alpha \vek{j}_{\text{vol}}\,.
\end{equation}

% ------------------------------------------------------------------------------
\section{Mass balances}
\label{sec:mass}

\paragraph{Mixture}
For every phase, including solid $\alpha = s$ and fluid $\alpha = f$, the
partial mass balance is
\begin{equation}
  \label{eq:phaseMass}
  \frac{\partial (\phi^\alpha \rho^\alpha_T) }{\partial t} +
  \nabla \cdot (\phi^\alpha \rho^\alpha_T \V^\alpha) = r^\alpha
\end{equation}
with the true phase density $\rho^\alpha_T$ and some reaction term $r^\alpha$.
Obviously, $\sum r^\alpha = 0$ is the condition that the total mass of the
mixture does not change.  As every phase is assumed to be intrinsically
incompressible, i.e. $\rho^\alpha_T$ is a constant, this equation reduces to
\begin{equation}
  \label{eq:phaseMass2}
  \frac{\partial \phi^\alpha}{\partial t} + \nabla \cdot 
  (\phi^\alpha \V^\alpha) = \frac{r^\alpha}{\rho_T^\alpha}\,.
\end{equation}
Now, we assume that $\phi^s + \phi^f \approx 1$, i.e. that the volume
fractions $\phi^\alpha$ are negligible. Since there will be no mass production
terms for the solid and fluid constituents, i.e. $r^s = r^f = 0$, we can sum
above equation for the solid and fluid in order to obtain
\begin{equation}
  \label{eq:mixtureMass}
  \nabla \cdot \left[ \phi^s \V^s + \phi^f \V^f  \right] = 
  \nabla \cdot \left[ \V^s + \vek{j}_{\text{vol}} \right] = 0\,.
\end{equation}

\paragraph{Solute}
The mass balance of the solute $\alpha$ can be derived
from~\eqref{eq:phaseMass}
by setting $\phi^\alpha \rho^\alpha_T = \rho^\alpha = M^\alpha \phi^f
c^\alpha$
and dividing by the constant $M^\alpha$ from the equation
\begin{equation}
  \label{eq:soluteMass1}
  \frac{\partial (\phi^f c^\alpha) }{\partial t} + \nabla \cdot \left[ \phi^f
    c^\alpha \V^\alpha \right] = \frac{r^\alpha}{M^\alpha}\,.
\end{equation}
Writing for the phase velocity $\V^\alpha = (\V^\alpha - \V^s) + \V^s$ and
replacing the difference by expression~\eqref{eq:aux1}, we get
\begin{equation}
  \label{eq:soluteMass}
  \frac{\partial( \phi^f c^\alpha )}{\partial t} - \nabla \cdot \left[
    \phi^f D^\alpha \nabla c^\alpha - \phi^f c^\alpha \left( \V^s +
      \frac{D^\alpha}{D^\alpha_0} (\V^f - \V^s) \right) \right] 
  = \frac{r^\alpha}{M^\alpha}\,.
\end{equation}

\section{Simplifications}
\label{sec:simple}

The complete system of equations is given by 
\begin{itemize}
\item the mixture momentum balance~\eqref{eq:momMix},
\item the mixture mass balance~\eqref{eq:mixtureMass}, and
\item the solute mass balance~\eqref{eq:soluteMass}\,.
\end{itemize}


The volumetric flux $\vek{j}_{\text{vol}}$ depends on gradients of the
concentrations and the permeability depends on the concentrations
themselves. Both dependencies lead to a fully coupled and highly non-linear
system of equations. In order to simplify the basic equations, we aim to
remove these dependencies. Effectively, this is achieved by saying that
 $D^\alpha \to D^\alpha_0$.
 This means physically that the diffusivity of every constituent $\alpha$ is
the same value in the mixture as in a free solution. Henceforth, we assume
that
\begin{equation}
  \label{eq:diffusivity}
  D^\alpha = D^\alpha_0\,.
\end{equation}

This simplification leads to the fact that $k_{\text{eff}} = k$ and we recover
Darcy's law
\begin{equation}
  \label{eq:darcy}
  \vek{j}_{\text{vol}} = - k \nabla p \,.
\end{equation}
Moreover, already employing this relation, the molecular flux becomes
\begin{equation}
  \label{eq:molecularFlux2}
  \vek{j}^\alpha = -\phi^f D^\alpha \nabla c^\alpha - c^\alpha k \nabla p\,.
\end{equation}

The fluid velocity $\V^f$ is now eliminated by means of the volumetric flux,
i.e. $\phi^f \V^f = \vek{j}_{\text{vol}} + \phi^f \V^s$, and based on the
simplified form~\eqref{eq:darcy} we get from~\eqref{eq:phaseMass2} for
$\alpha = f$ and $r^f = 0$
\begin{equation}
  \label{eq:fluidMass2}
  \frac{\partial \phi^f}{\partial t} + \nabla \cdot (\phi^f \V^s - k \nabla p) = 0\,.
\end{equation}

Consider now the balance of mass of the solute $\alpha$ as given
in~\eqref{eq:soluteMass} and simplified via~\eqref{eq:diffusivity}
\begin{equation} \label{eq:soluteMass2}
  \begin{aligned}
    \frac{ \partial(\phi^f c^\alpha)}{\partial t} - \nabla \cdot \left[ \phi^f
      D^\alpha \nabla c^\alpha
      - c^\alpha (\phi^f \V^s - k \nabla p) \right] &=\\
    c^\alpha \left[ \frac{\partial \phi^f}{\partial t} + \nabla \cdot (\phi
      \V^s - k \nabla p) \right] + \phi^f \frac{\partial c^\alpha}{\partial t}
    + \left[ \phi^f \V^s - k \nabla p \right] \cdot \nabla c^\alpha -
    \nabla \cdot \left[ \phi^f D^\alpha \nabla c^\alpha \right] &= \\
    \phi^f \frac{\partial c^\alpha}{\partial t} + \left[ \phi^f \V^s - k
      \nabla p \right] \cdot \nabla c^\alpha 
    - \nabla \cdot \left[ \phi^f D^\alpha \nabla c^\alpha \right] &= 
    \frac{r^\alpha}{M^\alpha}\,.
\end{aligned}
\end{equation}

% ------------------------------------------------------------------------------
\begin{comment}
Performing similar steps that lead to~\eqref{eq:molecularFlux2} for the
balance of mass of the solid phase, one gets
\begin{equation}
  \label{eq:solidMass}
  \rho^s_T \left[ \frac{\partial \phi^s}{\partial t} + \nabla \cdot (\phi^s
    \V^s) \right] + \phi^s \left[ \frac{\partial \rho_T^s}{\partial t} + \V^s
    \cdot \nabla \rho_T^s \right] = 0\,.
\end{equation}
In case of an incompressible solid material, the second term vanishes and the
first term has to be zero too. Adding now this first term to the fluid mass
balance~\eqref{eq:fluidMass2}, one gets
\begin{equation}
  \label{eq:mixtureMass2}
  \frac{ \partial (\phi^f + \phi^s) }{\partial t} + \nabla \cdot \left[
    \phi^f \V^s - k \nabla p + \phi^s \V^s \right] = 0\,.
\end{equation}
Based on the previous assumption that the solute volume fractions are
negligible, we have $\phi^f + \phi^s = 1$ and, therefore, 
\begin{equation}
  \label{eq:mixtureMass3}
  \nabla \cdot \left[ \V^s - k \nabla p \right] = 0\,.
\end{equation}

\end{comment}
% ------------------------------------------------------------------------------

Finally, the identification 
\begin{equation}
  \label{eq:solidVelocity}
  \V^s = \frac{\partial \U}{\partial t} = \partial_t \U
\end{equation}
closes the loop and connects back to the solid displacements. 

\subsection{Summary of simplified model} 

The simplifications made above are
\begin{itemize}
\item negligible solvent volume fractions $\phi^\alpha \ll \phi^f$ and
  incompressible solid and fluid constituents; used for
  the mixture mass balance~\eqref{eq:mixtureMass}
\item diffusivity $D^\alpha$ in the mixture does not differ from the
  diffusivity $D^\alpha_0$ in a free solution; allows to uncouple the system
  of equations. The same simplifications are obtained by assuming that 
  \begin{enumerate}
  \item The contribution of the concentration gradient $\nabla c^\alpha$ to
    the volumetric flux $\vek{j}_{\text{vol}}$ is negligible
  \item The contribution of the concentration $c^\alpha$ to the permeability
    is negligible
  \end{enumerate}
\item incompressible fluid constituent; leads to
  equation~\eqref{eq:fluidMass2} and thus the simplification of the solute
  mass balance~\eqref{eq:soluteMass2}
\end{itemize}

The final set of equations is obtained as the mixture momentum balance
equation~\eqref{eq:momMix}, which has not been altered due to any assumption,
the mixture mass balance~\eqref{eq:mixtureMass}, using the simplified volumetric
flux~\eqref{eq:darcy} and the solid velocity~\eqref{eq:solidVelocity}, and
finally the mass balances of every solute $\alpha$ as given
by~\eqref{eq:soluteMass2}
\begin{subequations}\label{eq:simpleModel}
  \begin{align} 
    -\nabla \cdot \vek{\sigma}(\U) + \nabla p &= \vek{f} \label{eq:momB} \\
    \nabla \cdot \partial_t \U - \nabla \cdot (k \nabla p) &=
    0 \label{eq:masB} \\
    \phi^f \frac{\partial c^\alpha}{\partial t} + 
    \left[ \phi^f \partial_t \U - k \nabla p \right] \cdot \nabla c^\alpha 
    - \nabla \cdot \left[ \phi^f D^\alpha \nabla c^\alpha \right] &=
    \frac{r^\alpha}{M^\alpha}\,.
    \label{eq::massBalpha}
  \end{align}
\end{subequations}
Note that equations~\eqref{eq:momB} and~\eqref{eq:masB} do not depend on
$c^\alpha$. This allows to decouple this system of equations. In every time
step, first the current solid displacement $\U$ and fluid pressure $p$ are
computed, before equation~\eqref{eq::massBalpha} is solved for every
solute~$\alpha$. It is evident, how the solid velocity $\partial_t \U$ and the
pressure $p$ constitute the apparent fluid velocity $\phi^f \V^f$, which in
turn determines the convective term of~\eqref{eq::massBalpha}. 


Moreover, the assumptions made so far are not restricted to small solid
deformations. In fact, system~\eqref{eq:simpleModel} should still valid be as
the spatial equations in a large deformation formulation. However, in that
case the fluid volume fraction will obey the relation
\begin{equation}
  \label{eq:volumeFraction}
  \phi^f = (1 - \phi^s) = 1 - \frac{\phi^s_0}{J} = 1 - \frac{1 - \phi^f_0}{J}\,,
\end{equation}
where $\phi^f_0$ is the initial volume fraction in reference configuration and
$J = \det \vek{F}$ the determinant of the deformation gradient of the solid
matrix. For clarity, it has to be remarked that above derivations assumes that
fluid and solid are incompressible. This is the so-called intrinsic
incompressiblity which refers to the case of a block of material being either
entirely solid or entirely fluid. Since the solid forms a skeleton with pores,
the solid phase will still be compressible and there is no restriction $J = 1$
as in incompressible elasticity.

\subsection{Weak form}
\label{sec:weakform}

The weak form of system~\eqref{eq:simpleModel} is obtained via a weighted
residual approach, i.e. the residuals of these equations will be multiplied by
admissible test fields $\delta \U$, $\delta p$ and $\delta c^\alpha$,
respectively, and integrated over the domain $\Omega$. Since the equations are
decoupled, first the poroelastic part~\eqref{eq:momB} and~\eqref{eq:masB} are
considered 
\begin{equation}
  \label{eq:weakF1}
  0 = 
  \int_\Omega \left( -\nabla \cdot \vek{\sigma}(\U) + \nabla p - \vek{f} \right) 
  \cdot \delta \U  \dif \x + 
  \int_\Omega \left( \nabla \cdot \partial_t \U - 
    \nabla \cdot (k \nabla p) \right) \delta p \dif \x \,.
\end{equation}
By means of integration by parts, the following expression is obtained
\begin{multline}
  \label{eq:weakF2}
  \int_\Omega \vek{\sigma}(\U) \colon \nabla (\delta \U) \dif \x - 
  \int_\Omega p (\nabla \cdot \delta \U) \dif \x -
  \int_\Gamma (\vek{\sigma}(\U) \cdot \vek{n}) \cdot \delta \U \dif s +
  \int_\Gamma (p \vek{n}) \cdot \delta \U \dif s \\
  + \int_\Omega (\nabla \cdot \partial_t \U) \delta p \dif \x +
  \int_\Omega (k \nabla p) \cdot \nabla (\delta p) \dif \x -
  \int_\Gamma (k \nabla p) \cdot \vek{n} \delta p \dif s =
  \int_\Omega \vek{f} \cdot \delta \U \dif \x
\end{multline}
with the outward unit normal vector $\vek{n}$.
The mass balance of the solute $\alpha$, on the other hand, leads to the weak
form 
\begin{multline}
  \label{eq:weakF3}
  \int_{\Omega} \phi^f (\partial_t c^\alpha) \delta c^\alpha \dif \x +
  \int_\Omega \left[ \phi^f \partial_t \U - k\nabla p \right] \cdot 
  (\nabla c^\alpha) \partial c^\alpha \dif \x   \\
  + \int_\Omega \left[ \phi^f D^\alpha \nabla c^\alpha\right] \cdot 
  \nabla \partial c^\alpha \dif \x - 
  \int_\Gamma \left[ \phi^f D^\alpha \nabla c^\alpha\right] \cdot \vek{n}
  (\delta c^\alpha) \dif x = 
  \frac{1}{M^\alpha} \int_\Omega r^\alpha \delta c^\alpha \,.
\end{multline}

Let the boundary $\Gamma = \partial \Omega$ be partitioned in a disjoint way
\begin{equation}
  \label{eq:boundaryPart}
  \Gamma = \Gamma^{\U}_D \cup \Gamma^{\U}_N = \Gamma^p_D \cup \Gamma^p_N
  = \Gamma^\alpha_D \cup \Gamma^\alpha_N\,.
\end{equation}
On each of these six boundary parts, the following boundary conditions are
prescribed 
\begin{subequations} \label{eq:BC}
  \begin{align}
    \U &= \bar{\U} &\x &\in \Gamma_D^{\U} \label{eq:BCuD} \\
    \vek{t} = \vek{\sigma}(\U) \cdot \vek{n} &= \bar{\vek{t}}
    &\x &\in \Gamma_N^{\U} \label{eq:BCuN} \\
    p &= \bar{p}  &\x &\in \Gamma_D^p \label{eq:BCpD} \\
    (k \nabla p) \cdot \vek{n} &= \bar{q}
    &\x &\in \Gamma_N^p \label{eq:BCpN} \\
    c^\alpha &= \bar{c}^\alpha &\x &\in \Gamma_D^\alpha \label{eq:BCaD} \\
    \left[\phi^f D^\alpha \nabla c^\alpha \right] \cdot \vek{n} &=
    \bar{d}^\alpha 
    &\x &\in \Gamma_N^\alpha \label{eq:BCaN} 
  \end{align}
\end{subequations}
Now the boundary partition~\eqref{eq:boundaryPart} and the
conditions~\eqref{eq:BC} are used for the weak forms~\eqref{eq:weakF2}
and~\eqref{eq:weakF3}. Here, the classic approach is pursued in which the
Dirichlet conditions~\eqref{eq:BCuD}, \eqref{eq:BCpD} and~\eqref{eq:BCaD} are
employed as essential conditions to the trial and test spaces. The Neumann
conditions \eqref{eq:BCuN}, \eqref{eq:BCpN} and~\eqref{eq:BCaN} are handled in
a natural way. For the poroelastic system~\eqref{eq:weakF2}, we obtain:
\emph{Find $(\U, p) \in V^{\U}_E \times V^p_E$ such that}
\begin{equation}
  \label{eq:wfPoro}
  \begin{aligned}
    a( \U, \delta \U) - b( p, \delta \U )&= \ell^{\U} (\delta \U) \\
    b( \delta p, \partial_t \U) + e( p, \delta p ) &= \ell^p (\delta p)
  \end{aligned}
\end{equation}
\emph{for all $(\delta \U, \delta p) \in V^{\U}_0 \times V^p_0$}.
For every solute $\alpha$ we formulate:
\emph{Find $c^\alpha \in V^\alpha_E$ such that}
\begin{equation}
  \label{eq:wfAlpha}
  m( \partial_t c^\alpha, \delta c^\alpha) + c( \partial_t \U, p; c^\alpha, \delta
  c^\alpha ) + k( c^\alpha, \delta c^\alpha ) = \ell^\alpha (\delta c^\alpha) 
\end{equation}
\emph{for all $\delta c^\alpha \in V^\alpha_0$}.
The spaces in these variational principles are the standard Sobolev spaces
with essential boundary conditions (subscript $E$) and zero essential boundary
conditions (subscript $0$). The introduced bilinear and linear forms are
defined as follows
\begin{subequations} \label{eq:forms}
  \begin{align}
      a(\U,\V) &= \int_\Omega \vek{\sigma}(\U) \colon \nabla \V \dif \x 
      \label{eq:formsA} \\
      b(q,\V) &= \int_\Omega (\nabla \cdot \V) q \dif \x
      \label{eq:formsB} \\
      e(p,q) &= \int_\Omega (k \nabla p) \cdot \nabla q \dif \x
      \label{eq:formsE} \\
      m(c,d) &= \int_\Omega \phi^f c \,d \dif \x
      \label{eq:formsM} \\
      c(\V,q; c, d) &= \int_\Omega \left[ \phi^f \V - k \nabla q \right] \cdot
      (\nabla c) \, d \dif \x
      \label{eq:formsC} \\
      k(c, d) &= \int_\Omega \left[ \phi^f D^\alpha \nabla c\right] \cdot 
      \nabla d \dif \x
      \label{eq:formsK} \\
      \ell^{\U}(\V) &= \int_\Omega \vek{f} \cdot \V \dif \x +
      \int_{\Gamma_N^{\U}} \bar{\vek{t}} \cdot \V \dif s
      \label{eq:formsLU} \\
      \ell^p(q) &= \int_{\Gamma^p_N} \bar{q} \, q \dif s
      \label{eq:formsLP} \\
      \ell^\alpha(d) &= 
      \frac{1}{M^\alpha} \int_\Omega r^\alpha \, d \dif \x +
      \int_{\Gamma^\alpha_N} \bar{d}^\alpha \, d \dif s
      \label{eq:formsLA} 
  \end{align}
\end{subequations}

\section{Reaction kinetics}
\label{sec:reaction}
Let one chemical reactions among the solutes $\alpha \neq s, f$ take
place. For notational simplicity let, $\alpha \in \{ A_k \}_{k = 1}^K$ be the
reactants and $\alpha \in \{ B_\ell \}_{\ell=1}^L$ be the products of the
chemical reaction
\begin{equation}
  \label{eq:reaction}
  \sum_{k=1}^K \gamma_k A_k \xrightleftharpoons[K^-]{K^+} 
  \sum_{\ell=1}^L \delta_\ell B_\ell \,.
\end{equation}
with the composition $K^+$ and decomposition $K^-$ rates. Here,
$\gamma_k$ and $\delta_\ell$ are the stoichiometric coefficients of the
reaction. 
The mass production term of the solute $\alpha$ is now
\begin{equation}
  \label{eq:massProduct}
  r^\alpha = M^\alpha \left( \delta_\alpha - \gamma_\alpha \right) 
  \left[ K^+ \prod_{k} [A_k]^{\gamma_k} - 
    K^- \prod_{\ell} [B_\ell]^{\delta_\ell} \right]\,.
\end{equation}
Here, the notation $[ C ]$ denotes the effectively available concentration of
the solute $C$ and is given as
\begin{equation}
  \label{eq:effectiveConcentration}
  [C] = \frac{\rho^C}{M^C} = \phi^f c^C \,.
\end{equation}

\paragraph{Simple reaction}
For the simple reaction
\begin{equation}
  \label{eq:simpleReaction}
  1 \cdot A + 1 \cdot B \xrightleftharpoons[K^-]{K^+}  1 \cdot C
\end{equation}
the mass production terms evaluate to
\begin{equation}
  \label{eq:simpleMassProduct}
  \begin{aligned}
    r^A &= -M^A \left( K^+ (\phi^f c^A)^1 (\phi^f c^B)^1 - K^- (\phi^f c^C)^1
    \right) = -M^A \phi^f \left( K^+ \phi^f c^A c^B - K^- c^C \right)\\
    r^B &=  -M^B \phi^f \left( K^+ \phi^f c^A c^B - K^- c^C \right)\\
    r^C &=  +M^C \phi^f \left( K^+ \phi^f c^A c^B - K^- c^C \right)
  \end{aligned}
\end{equation}
The condition $\sum r^\alpha = 0$ gives the condition
\begin{equation}
  \label{eq:control}
  M^A + M^B = M^C
\end{equation}
which clearly states the conservation of molecular masses during the
elementary reaction~\eqref{eq:simpleReaction}. 

\section{Example}
\label{sec:ex}

\begin{figure}[htbp]
  \centering
  \begin{tikzpicture}[scale=1.0]
    \draw (0,0) rectangle (4,4);
    \draw (-.4,2) node{$p_1$};
    \draw (0.4,2) node{$c^\alpha_0$};
    \draw (4.4,2) node{$p_2$};
    \draw[->,thick] (0,0) -- (1,0);
    \draw (1,.4) node{$x_1$};
    \draw[->,thick] (0,0) -- (0,1);
    \draw (.4,1) node{$x_2$};
  \end{tikzpicture}
  \caption{Trivial example on $\Omega = (0,1)^2$}
  \label{fig:trivial}
\end{figure}

Initial conditions
\begin{equation}
  \label{eq:initial}
  \U = 0\,,\quad p=0\,, \quad c^\alpha = c^\alpha_0
\end{equation}
Boundary conditions
\begin{equation}
  \label{eq:boundary}
  \begin{aligned}
    u_1 &=0\,, \quad p = p_1\,, \quad c^\alpha = c^\alpha_0 &x_1 &= 0\\
    u_2 &=0  &x_2 &= 0\\
    u_1 &=0\,, \quad p = p_1  &x_1 &= 1\\
    u_2 &=0  &x_2 &= 1\\
  \end{aligned}
\end{equation}
all other boundary conditions are homogeneous Neumann conditions.
The concentration boundary condition is equal to the initial concentration,
see below. The pressure boundary values are chosen as follows
\begin{equation}
  \label{eq:pressureBC}
  \begin{aligned}
    &\text{no pressure gradient:}\qquad &p_1 &=p_2 = 1 \\
    &\text{with pressure gradient:}    &p_1  &=2 \,,\quad p_2 = 1
  \end{aligned}
\end{equation}

Material parameter
\begin{equation}
  \label{eq:material}
  \begin{aligned}
    E &= 1000 \qquad &\nu &= 0.3 \qquad   &k &= 1.0\\
    \phi^f &= 0.5 &D &= 0.2
  \end{aligned}
\end{equation}

Chemical reaction
\begin{equation}
  \label{eq:react}
  A + B \xrightleftharpoons[K^-]{K^+}   C
\end{equation}
with the reaction rates
\begin{equation}
  \label{eq:rates}
  K^+ = 3 \qquad K^- = 1
\end{equation}
and the initial conditions
\begin{equation}
  \label{eq:initalConc}
  c^A_0 = 0.5 \qquad c^B_0 = 0.2 \qquad c^C_0 = 0.3
\end{equation}


\begin{figure}[htbp]
  \centering
  \includegraphics[scale=0.8]{concentration}
  \caption{Temporal variation of the concentrations at mid point $\x = (0.5,0.5)$}
  \label{fig:result}
\end{figure}

\bibliographystyle{plain}
\bibliography{./multiPhasic.bib}
\end{document}



%%% Local Variables: 
%%% mode: latex
%%% TeX-master: t
%%% End: 
